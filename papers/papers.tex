
\documentclass{report}

\input{~/.config/latex_template/preamble.tex}
\input{~/.config/latex_template/macros.tex}
\input{~/.config/latex_template/letterfonts.tex}

\title{\Huge{{Machine Learning in C}}}
\author{\huge{Luka Flores}}

\begin{document}

\maketitle
\newpage% or \cleardoublepage
% \pdfbookmark[<level>]{<title>}{<dest>}
\pdfbookmark[section]{\contentsname}{toc}
\tableofcontents
\pagebreak


\chapter{Introduction} % (fold)
\label{chap:Introduction}

\section{Preface} % (fold)
\label{sec:Preface}

% section Preface (end)



\section{Defintions} % (fold)
\label{sec:Defintions}

\subsubsection{Cost or Loss Function} % (fold)
\label{sec:Cost or Loss Function}
% subsubsection Cost or Loss Function (end)


% section Defintions (end)
% chapter Introduction (end)



\chapter{Basic Example of Machine Learning} % (fold)
\label{chap:Basic Example of Machine Learning}


To start we will try to train a model that will correctly predict this set.

\begin{code}[language=c]{Data Set}
float train[][2] = {
    {0,0},
    {1,2},
    {2,4},
    {3,6},
    {4,8},
    {5,10},
}
\end{code}


If we think of the left value in each row as the input and the right as the output. The astute would notice a pattern.

\begin{align*}
    \text{output} &= 2 \times \text{ input} \\
\end{align*}

We can turn this into a \textbf{ mathematical equation:} 
\begin{align*}
    \qquad y &= w x 
\end{align*}

Where, $w = 2$


\subsubsection{Goal:} % (fold)
\label{sec:Goal:}

Our Goal is the create a model, that will train on the data to find $w$. As a result it will be able to predict future the outputs given an input.
% subsubsection Goal: (end)

\subsection{Method} % (fold)
\label{sub:Method}

To train the model we it needs a method for the model to produce numbers, then we will create a lost or cost function (\ref{sec:Cost or Loss Function}) to assess how accurate the model's guess is.

For this simple example we will create a function that makes random guesses



\begin{code}[language=c]{Random Function in C}
// rand_float produces a random number between 0 and 1 
float rand_float(void) {
    return (float) rand() / (float) RAND_MAX; 
}

int main(){
    //srand(time(0)); 
    srand(12);
}
\end{code}

$\verb|srand()|$ is a way to adjust the random see that is used by $\verb|rand()|$. If the value inside $\verb|srand()|$ is constant the seed will remain constant, and therefore then random number will not change every time it is run. To have a changing random number we can use $\verb|time(0)|$ a function that produces the current time.  


$\verb|srand(time(0))|$ will consistently produce random seeds which in turn will produce random numbers in $\verb|rand_float()|$


In the case of training we will limit the seed to 1 seed so we can provide a constant value within $\verb|srand(12)|$ 12 as an example


%%TSODING EP 1 : 20:03



% subsection Method (end)
% chapter Basic Example of Machine Learning (end)





\end{document}

